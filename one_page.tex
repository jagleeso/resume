% resume.tex
%
% (c) 2002 Matthew Boedicker <mboedick@mboedick.org> (original author) http://mboedick.org
% (c) 2003-2007 David J. Grant <davidgrant-at-gmail.com> http://www.davidgrant.ca
%
% This work is licensed under the Creative Commons Attribution-ShareAlike 3.0 Unported License. To view a copy of this license, visit http://creativecommons.org/licenses/by-sa/3.0/ or send a letter to Creative Commons, 171 Second Street, Suite 300, San Francisco, California, 94105, USA.
%\documentclass[letterpaper,11pt]{article}
\documentclass[letterpaper,11pt]{article}

%-----------------------------------------------------------
%Margin setup

\setlength{\voffset}{0.1in}
\setlength{\paperwidth}{8.5in}
\setlength{\paperheight}{11in}
\setlength{\headheight}{0in}
\setlength{\headsep}{0in}
\setlength{\textheight}{11in}
\setlength{\textheight}{9.5in}
% \setlength{\topmargin}{-0.50in}
\setlength{\textwidth}{7in}
\setlength{\topskip}{0in}

% % \setlength{\topmargin}{-1.0in}
% \setlength{\topmargin}{-0.25in}
% % \setlength{\bottommargin}{-0.50in}
% \setlength{\oddsidemargin}{-0.25in}
% \setlength{\evensidemargin}{-0.25in}

%% FORMAT %%
% https://www.sgs.utoronto.ca/awards-funding/scholarships-awards/ontario-graduate-scholarship-application-instructions/?highlight=OGS#section_5
% Body text in a minimum 12 pt Arial, Times New Roman, or similar font
% Single-spaced, with no more than 6 lines of type per inch
% All margins set at a minimum of 3/4″ (1.87 cm)

\usepackage{enumitem}
\usepackage{anysize}
% \marginsize{left}{right}{top}{bottom}
% OGS / Bell scholarship: All margins set at a minimum of 3/4" (1.87 cm)
%\marginsize{0.75in}{0.75in}{0.25in}{0.25in}
\marginsize{0.75in}{0.75in}{0.75in}{0.75in}

%%
%% ATC 2020 template (Usenix 2019)
%%
% refs and bib
\usepackage{cite}               % order multiple entries in \cite{...}
\usepackage{breakurl}           % break too-long urls in refs
\usepackage{url}                % allow \url in bibtex for clickable links
\usepackage{xcolor}             % color definitions, to be use for...
\usepackage[]{hyperref}         % ...clickable refs within pdf...
\hypersetup{                    % ...like so
  colorlinks,
  linkcolor={green!80!black},
  citecolor={red!70!black},
  urlcolor={blue!70!black}
}

%\usepackage{amssymb}
%\usepackage{amsmath}
%\usepackage{amsthm}
%\usepackage{xparse}
%\usepackage{tikz}

% \asPercent
\usepackage{pgfmath}
\usepackage{siunitx}

%-----------------------------------------------------------
%\usepackage{fullpage}
% \usepackage{shading}
\usepackage{bold-extra}
%\textheight=9.0in
\pagestyle{empty}
\raggedbottom
\raggedright
\setlength{\tabcolsep}{0in}

%-----------------------------------------------------------
%Custom commands

\newcommand{\company}[1]{
    \textbf{#1}
}

\newcommand{\heading}[1]{
    \textsc{\textbf{#1}}
}

\newcommand{\asPercent}[1]{\pgfmathparse{100*#1}\num[round-mode=places,round-precision=1]{\pgfmathresult}\%}

% title for the root sections (experience, education, etc) of the resume
\newcommand*\resheading[1]{\subsection*{\heading{#1}}\vspace{0.3em}\nopagebreak[4]}
% \newcommand\resheading[1]{\vspace{-0.3em}}

\newcommand{\resitem}[1]{\item #1 \vspace{-2pt}}
% \newcommand{\resheading}[1]{{\large \parashade[.95]{roundcorners}{\textbf{#1 \vphantom{p\^{E}}}}}}
\newcommand{\ressubheading}[4]{
\begin{tabular*}{6.5in}{l@{\extracolsep{\fill}}r}
    
		\company{#1} & #2 \\
		\textit{#3} & \textit{#4} \\
\end{tabular*}\vspace{-6pt}}
\newcommand{\mscphdheading}[6]{
\begin{tabular*}{6.5in}{l@{\extracolsep{\fill}}r}
    
		\company{#1} & #2 \\
		\textit{#3} & \textit{#4} \\
		\textit{#5} & \textit{#6} \\
\end{tabular*}\vspace{-6pt}}
\newcommand{\sickkids}[6]{
\begin{tabular*}{6.5in}{l@{\extracolsep{\fill}}r}
		\company{#1} & #2 \\
		\textit{#3} & \textit{#4} \\
		 & \textit{#5} \\
		 & \textit{#6} \\
\end{tabular*}\vspace{-6pt}}
% \newcommand{\ressubheadingnodate}[4]{
% \begin{tabular*}{6.5in}{l@{\extracolsep{\fill}}r}
% 		\textbf{#1} \\
% \end{tabular*}\vspace{-6pt}}
\newcommand{\ressubheadingnodate}[1]{
		% \textbf{#1} \\
		#1 \\
}
%-----------------------------------------------------------


\begin{document}

% \begin{tabular*}{7in}{l@{\extracolsep{\fill}}r}
% \textbf{\Large James Gleeson}  & 647-298-9193\\
% 499 Martha Street & jagleeso@gmail.com \\
% Burlington, ON L7R 2R1 \\
% \end{tabular*}
% \\

\begin{tabular*}{7in}{l@{\extracolsep{\fill}}r}
\textbf{\Large James Gleeson -- Plan of Study}  &  May 13, 2021\\
\textbf{Research area:} Software & 647-298-9193, jagleeso@gmail.com \\
\end{tabular*}
\\

\hrule

\vspace{0.1in}


%\resheading{Plan of Study}
% \heading{Plan of Study - [Research Area: Software]}
% \heading{Plan of Study}

\setlength{\parindent}{1.5em}

There have been several key enablers to advances in artificial intelligence applications such as image recognition.
\textbf{(1) Deep convolutional architectures} exploit the compositional hierarchy in which features are learned by backpropagation, with earlier layer features (e.g., edges) being used to detect more high-level features (e.g., eyes and mouths in a face)~\cite{lecun2015deep}.
In 2012, Alex Krizhevsky~\cite{krizhevsky2012imagenet} showed that training on a \textbf{(2) large data set} of 1.2 million high-resolution images scoured from the web allowed them to halve the top-5 error rate of the best competing image recognition models at the time.
Finally, to enable efficient training of large models on large data sets, researchers have \textbf{(3) scaled up computation} by repurposing GPU accelerators originally designed for accelerating graphics rendering to instead act as massive compute accelerators for evaluating neural network operations like convolutions and matrix multiplications.  In my research, I have focused on how we can continue \textbf{(3) scaling up computation} when applying deep neural networks to emerging application domains, such as Reinforcement Learning (RL).

% In particular, advances in deep learning have now begun advancing the state-of-the-art of Reinforcement Learning (RL).
%~\cite{mnih2016asynchronous,wang2016sample,wu2017scalable,lillicrap2015continuous,ho2016generative,andrychowicz2017hindsight,schulman2017proximal,schulman2015trust}
In the past decade, RL algorithms achieved and sometimes surpassed human performance in increasingly complex tasks, 
% ranging from simple games like Atari~\cite{mnih2015human}, 
including board-games with intractably large state-spaces such as Go~\cite{silver2017mastering}, and complex multiplayer strategy games like DotA 2~\cite{OpenAI_dota}.
RL algorithms are now being explored as candidate algorithms in increasingly complex industrial domains, including robotics~\cite{brockman2016openai,kober2013reinforcement}, autonomous driving~\cite{dosovitskiy2017carla,sallab2016end}, data center management~\cite{datacenterRL}, and drone applications~\cite{krishnan2019air}.
Despite their promise, RL models are notoriously slow to train.
% , with developers having to explore an enormous selection of RL algorithms and many hyperparameters for each such algorithm.
In my latest publication, I investigated the root cause of long training times in RL by building an open-source profiling tool RL-Scope~\cite{gleeson2021rlscope}, and used it to survey RL across three major workload dimensions: RL algorithm, simulator, and deep learning (DL) framework (e.g., PyTorch and TensorFlow).
I discovered that \emph{all} surveyed RL workloads have fundamental structural differences from supervised learning workloads that leads to chronic GPU underutilization ($< 14\%$): comparatively smaller neural networks, training data collected at runtime, and high-level language code gluing together the core training loop.  
RL-Scope's diagnosis motivates two directions of future research to reduce total training time by increasing GPU utilization of RL workloads:

\begin{enumerate}[leftmargin=*]
	\item \textbf{Efficient parallel model training:} 
            Many RL algorithms and many hyperparameters must be explored when applying RL to a problem, which means many models must be trained by ML researchers.  
            Given that a single model cannot fully occupy the GPU resources, we can increase GPU utilization by training multiple models in \emph{parallel} on the same GPU.  
            However, na\"ive parallel training can quickly exhaust limited GPU memory resources.  
            To address this challenge, I will explore fusing operators across multiple different model instances. 
            This fusion optimization will have the same computational benefits of increasing the minibatch size of a single model,
            but will avoid limitations in batch size scaling (e.g., divergence) that would otherwise limit compute scaling.

        % regardless of application domain, simulation time on the CPU is always a large training bottleneck, accounting 
        % for at least 30.2\% of training time, with robotics workloads being substantially more simulation-bound than 
        % than other RL workloads with as much as 70.6\% in simulation.  
        % Unfortunately, prior work focuses mostly on 
        % GPU optimizations which will be unable to substantially reduce RL training time.

        \item \textbf{Integration of DL frameworks with simulator libraries:}
            Each additional model trained in parallel will also require additional simulator instances, requiring greater simulation throughput. 
            To address this challenge, I will explore cross-stack optimizations that integrate DL frameworks with common simulator libraries to reduce overheads that exist today.
            First, shared memory optimizations can reduce data serialization overhead that exists in popular scale-up distributed RL frameworks like Ray~\cite{ray}; for example, a single copy of neural network weights can be stored once in GPU memory across multiple inference calls and backpropagation calls.
            Next, GPU kernels for inference can be fused with GPU kernels for physics in simulator libraries to increase the arithmetic intensity of RL workloads and reduce GPU kernel launch overheads.


        % scale-up multi-process RL workloads attempt to increase GPU hardware parallelism by parallelizing inference 
        % operations.  
        % However, developers use common off-the-shelf profiling tools that misleadingly only tell them how 
        % often the GPU is in use, but provide zero indication of the true GPU hardware occupancy achieved by the workload.

\end{enumerate}

\noindent
Upon publication, I will integrate these optimizations into the popular open-source framework Ray~\cite{ray} to benefit the RL research community, and democratize high throughput training of RL workloads.


% Despite their promise, RL models are notoriously slow to train, with developers having to explore an enormous selection of RL algorithm variants and hyperparameters.
% It is essential to reduce training time to allow ML researchers to iterate quickly when applying RL to emerging industrial domains.
% To combat this challenge in my research, I have created a tool called RL-Scope that can discover and diagnose performance bottlenecks in RL training workloads, allowing ML and systems researchers to fix the root cause of performance anomalies.
% %
% In contrast to prior work that is focused on traditional supervised learning workloads which are primarily GPU-bound, RL-Scope is able to provide cross-stack insights spanning both GPU-bound neural network operations and CPU-bound simulation code both within low-level framework code (e.g., TensorFlow) and high-level Python code that ML developers interface with directly.
% %
% I used RL-Scope to survey the state-of-the-art RL workloads, and made two major discoveries about RL workloads that will impact future RL research:
% %spanning domains such as robotics and drone applications, and covering scale-up RL workloads that try maximize their use of hardware parallelism.
% \begin{enumerate}
% 	\item \textbf{(1) CPU time is non-negligible in \textit{all} RL workloads:} 
%         regardless of application domain, simulation time on the CPU is always a large training bottleneck, accounting 
%         for at least 30.2\% of training time, with robotics workloads being substantially more simulation-bound than 
%         than other RL workloads with as much as 70.6\% in simulation.  
%         Unfortunately, prior work focuses mostly on 
%         GPU optimizations which will be unable to substantially reduce RL training time.
%
%         \item \textbf{(2) Scale-up RL workloads are poorly optimized:}
%         scale-up multi-process RL workloads attempt to increase GPU hardware parallelism by parallelizing inference 
%         operations.  
%         However, developers use common off-the-shelf profiling tools that misleadingly only tell them how 
%         often the GPU is in use, but provide zero indication of the true GPU hardware occupancy achieved by the workload.
%
% \end{enumerate}
% %
% Upon publication, I will open source RL-Scope to positively impact both ML researchers and ML practitioners by helping them locate and eliminate performance bottlenecks in RL training workloads. 
% In my future work, I will eliminate the bottlenecks I've discovered in scale-up RL workloads by integrating a high-performance inference server (e.g., TensorRT~\cite{tensorrt}) into the RL training pipeline.

\bibliographystyle{plain}
% \bibliography{\jobname}

\pagebreak
\resheading{References}
%\heading{References}
\begingroup
\renewcommand{\section}[2]{}%
%\renewcommand{\chapter}[2]{}% for other classes
\bibliography{references}
\endgroup


\end{document}
